% polysi.tex

%%%%%%%%%%%%%%%%%%%%
\begin{frame}{}
	\begin{center}
		基于依赖图的 SI 刻画定理~\ncite{AnalysingSI:JACM2018}.

		\vspace{0.50cm}
		\begin{theorem}[定理 4.1 of~\ncite{AnalysingSI:JACM2018}]
			一个历史执行满足 \textup{SI} 当且仅当它存在一个依赖图 (dependency graph),
			该依赖图无环或者环中至少包含两条相邻的 $\RW$ 边。
		\end{theorem}
	\end{center}
\end{frame}
%%%%%%%%%%%%%%%%%%%%

%%%%%%%%%%%%%%%%%%%%
\begin{frame}{}
  \begin{center}
		% \only<4>{
		% 	% $T_{0} \rel{\WR} T_{A} \land T_{0} \rel{\WW} T_{B}
		% 	%   \implies T_{A} \rel{\RW} T_{B}$
		% 	% $T_{A}$ reads from $T_{0}$ which is overwritten by $T_{B}$
		% }
		% \only<5>{
		% 	% $T_{0} \rel{\WR} T_{B} \land T_{0} \rel{\WW} T_{A}
		% 	%   \implies T_{A} \rel{\RW} T_{A}$
		% 	% $T_{B}$ reads from $T_{0}$ which is overwritten by $T_{A}$
		% }
		\only<7->{
			\red{$\boxed{\text{假设}\;\; T_{A} \rel{\WW} T_{B}}$}
		}

		\vspace{0.20cm}
		{\input{tikz/banking-lost-update-depgraph-tikz}}
		\vspace{0.20cm}

		% \only<2>{
		% 	$\WR$: ``write-read'' dependency capturing the ``read-from'' relation
		% }
		% \only<3>{
		% 	$\WW$: ``write-write'' dependency capturing the version order on $\acct$
		% }
		% \only<4-5>{
		% 	$\RW$: ``read-write'' dependency
		% }
		\only<6>{
			SI \blue{允许}环 $T_{A} \rel{\RW} T_{B} \rel{\RW} T_{A}$
		}
		\only<7>{
			SI \red{不允许}环 $T_{A} \rel{\WW} T_{B} \rel{\RW} T_{A}$
		}
  \end{center}
\end{frame}
%%%%%%%%%%%%%%%%%%%%

%%%%%%%%%%%%%%%%%%%%
\begin{frame}{}
	\begin{center}
		\uncover<2->{
			\red{$\boxed{\text{\it 假设}\;\; T_{B} \rel{\WW} T_{A}}$}
		}

		\vspace{0.20cm}
		{\input{tikz/banking-lost-update-depgraph-ww-tbta-tikz}}
		\vspace{0.20cm}

		\uncover<2>{
			SI \red{不允许}环 $T_{B} \rel{\WW} T_{A} \rel{\RW} T_{B}$
		}
	\end{center}
\end{frame}
%%%%%%%%%%%%%%%%%%%%

%%%%%%%%%%%%%%%%%%%%
\begin{frame}{}
  \begin{center}
		$T_{A} \rel{\WW} T_{B}$
		与 $T_{B} \rel{\WW} T_{A}$ 两种情况都会导致违反 SI 的环。

		\vspace{0.20cm}
		\fig{width = 0.65\textwidth}{figs/banking-lost-update-wr}
		\vspace{0.20cm}

		因此, 该执行历史不满足 SI。
  \end{center}
\end{frame}
%%%%%%%%%%%%%%%%%%%%

% %%%%%%%%%%%%%%%%%%%%
% \begin{frame}{}
% 	\begin{theorem}[Equivalence of Theorem 4.1 of~\ncite{AnalysingSI:JACM2018}]
% 		Informally, a history satisfies SI if and only if \\[3pt]
% 		\teal{there exists} a \red{dependency graph} $\G$ for it such that \\[3pt]
% 		\cyan{the induced graph of $\G$} ${\boxed{((\SO_{\G} \cup \WR_{\G} \cup \WW_{\G}) \comp \RW_{\G}?)}}$ \text{\it is acyclic}.
% 	\end{theorem}
% \end{frame}
% %%%%%%%%%%%%%%%%%%%%

% %%%%%%%%%%%%%%%%%%%%
% \begin{frame}{Dependency Graph based Characterization of SI}
% 	\[
% 		\text{\it induced graph}\; {\boxed{((\SO_{\G} \cup \WR_{\G} \cup \WW_{\G}) \comp \RW_{\G}?)}} \text{\it\; for $\G$}
% 	\]

% 	\begin{center}
% 		\resizebox{0.60\textwidth}{!}{\input{tikz/banking-lost-update-depgraph-theorem-tikz.tex}}
% 		\vspace{0.30cm}

% 		\uncover<2->{
% 			first composing ($\comp$) the $\SO$/$\WR$/$\WW$ edges with the $\RW$ edges
% 		}

% 		\vspace{0.20cm}
% 		\uncover<3>{
% 			then deleting all the $\RW$ edges
% 		}
% 	\end{center}
% \end{frame}
% %%%%%%%%%%%%%%%%%%%%

%%%%%%%%%%%%%%%%%%%%
\begin{frame}{}
	\begin{center}
		Polygraph: 在一个结构中表达依赖图的所有可能性
	\end{center}

	\vspace{-0.20cm}
	\begin{columns}[c]
		\column{0.50\textwidth}
			\fig{width = 0.80\textwidth}{figs/banking-lost-update-depgraph}
		\column{0.50\textwidth}
			\fig{width = 0.80\textwidth}{figs/banking-lost-update-depgraph-ww-tbta}
	\end{columns}

	\vspace{-0.20cm}
	\begin{center}
		\pause
		\fig{width = 0.50\textwidth}{figs/banking-lost-update-polygraph}
		polygraph:
		$\tuple{\cyan{\eithervar} \triangleq \set{T_{A} \rel{\WW} T_{B}},
				\cyan{\orvar} \triangleq \set{T_{B} \rel{\WW} T_{A}, T_{A}' \rel{\RW} T_{A}}}
		$
	\end{center}
\end{frame}
%%%%%%%%%%%%%%%%%%%%

%%%%%%%%%%%%%%%%%%%%
\begin{frame}{}
	\begin{center}
		\fig{width = 0.90\textwidth}{figs/polysi-checker-pruning-encoding-solving}
	\end{center}
\end{frame}
%%%%%%%%%%%%%%%%%%%%

%%%%%%%%%%%%%%%%%%%%
\begin{frame}{}
	\begin{center}
		\resizebox{0.90\textwidth}{!}{\input{tikz/polysi-alg-tikz}}

		\vspace{0.30cm}
		\only<1>{
			待确定 $T_{0}$, $T_{1}$, and $T_{5}$ (on $\keyxvar$)
			以及 $T_{0}$ and $T_{2}$ (on $\keyyvar$) 之间的 $\WW$ 序
		}

		\only<2>{
			$T_{5} \rel{\WW(\keyxvar)} T_{0}$ 情况可以被剪枝
			($T_{0} \rel{\SO} T_{5} \rel{\WW(\keyxvar)} T_{0}$)
		}

		\only<4>{
			$T_{0} \rel{\WW(\keyxvar)} T_{5}$ 情况确定了
		}

		\only<6>{
			$T_{1} \rel{\WW(\keyxvar)} T_{0}$ 情况可以被剪枝
			($T_{3} \rel{\RW(\keyxvar)} T_{0} \rel{\WR(\keyyvar)} T_{3}$)
		}

		\only<8>{
			$T_{0} \rel{\WW(\keyxvar)} T_{1}$ 情况确定了
		}

		\only<10>{
			$T_{2} \rel{\WW(\keyyvar)} T_{0}$ 被剪枝,
			$T_{0} \rel{\WW(\keyyvar)} T_{2}$ 确定了
		}
		\only<12>{
			$T_{1}$ 与 $T_{5}$ 之间的 $\WW$ 序尚未确定
		}
	\end{center}
\end{frame}
%%%%%%%%%%%%%%%%%%%%

%%%%%%%%%%%%%%%%%%%%
\begin{frame}{}
	\vspace{-0.50cm}
	\[\tuple{
		\uncover<2->{\purple{\eithervar} = \set{T_{1} \rel{\WW(\keyxvar)} T_{5},
			T_{3} \rel{\RW(\keyxvar)} T_{5}}},
		\uncover<2->{\violet{\orvar} = \set{T_{5} \rel{\WW(\keyxvar)} T_{1}}}
	}\]

	\vspace{-0.30cm}
	\begin{center}
		\resizebox{0.80\textwidth}{!}{\input{tikz/polysi-alg-encoding-tikz}}
	\end{center}
	\vspace{-0.80cm}

	\uncover<3->{
		\[
			\underbrace{\purple{(\BV_{1,5} \land \BV_{3,5} \land \lnot \BV_{5,1})}}_{\purple{\eithervar}} \lor
			\underbrace{\violet{(\BV_{5,1} \land \lnot \BV_{1,5} \land \lnot \BV_{3,5})}}_{\violet{\orvar}}
		\]
	}
\end{frame}
%%%%%%%%%%%%%%%%%%%%

%%%%%%%%%%%%%%%%%%%%
% \begin{frame}{\textsc{PolySI}: A Running Example}
% 	\vspace{-0.50cm}
% 	\[
% 		\cyan{\text{induced graph}}\; \cyan{\mathcal{I}}\;
% 		  {\boxed{((\SO_{\G} \cup \WR_{\G} \cup \WW_{\G}) \comp \RW_{\G}?)}}
% 	\]

%   \begin{columns}
% 		\column{0.50\textwidth}
% 			\fig{width = 1.00\textwidth}{figs/polysi-alg-final}
% 		\column{0.50\textwidth}
% 			\fig{width = 1.00\textwidth}{figs/polysi-alg-encoding}
% 	\end{columns}

% 	\vspace{0.50cm}
% 	\uncover<2->{
% 	\[
% 		T_{1} \rel{\WR} T_{3} \rel{\RW} T_{5}:\;
% 		  \BV_{1,5}^{\cyan{\;\mathcal{I}}} = \BV_{1,3} \;\land\; \BV_{3,5} \quad
% 	\]

% 	\vspace{-0.20cm}
% 	\begin{center}
% 		The presence of the edge $T_{1} \to T_{5}$ in the induced graph $\mathcal{I}$ \\[2pt]
% 		depends on that of the edges $T_{1} \to T_{3}$ and $T_{3} \to T_{5}$ in the polygraph.
% 	\end{center}
% 	}
% 	% \vspace{-0.30cm}
% 	% \uncover<2->{
% 	% \[
% 	% 	T_{1} \rel{\WR} T_{3} \rel{\RW} T_{2}:\;
% 	% 	  \BV_{1,2}^{\cyan{\;\mathcal{I}}} = \BV_{1,3} \;\land\; \BV_{3,2} \quad
% 	% \]
% 	% }
% \end{frame}
% %%%%%%%%%%%%%%%%%%%%

% %%%%%%%%%%%%%%%%%%%%
% \begin{frame}{}
% 	\begin{center}
% 		Feed the SAT formula into the \blue{MonoSAT} solver~\ncite{MonoSAT:AAAI2015} \\[2pt]
% 		which is optimized for \purple{\it cycle detection}.

% 		\vspace{0.20cm}
% 		\fig{width = 0.40\textwidth}{figs/sat-solver}
% 	\end{center}
% \end{frame}
% %%%%%%%%%%%%%%%%%%%%

%%%%%%%%%%%%%%%%%%%%
\begin{frame}{}
	\begin{center}
		\fig{width = 0.50\textwidth}{figs/polysi-alg-cycle}

		\vspace{0.20cm}
		MonoSAT~\ncite{MonoSAT:AAAI2015} 求解器找到了违反 SI 的环~\footnote{
			在交给 MonoSAT 之前, 还需要做一些编码上的转换工作, 此处从略。
		}
	\end{center}
\end{frame}
%%%%%%%%%%%%%%%%%%%%

%%%%%%%%%%%%%%%%%%%%
\begin{frame}{}
	\begin{center}
		\fig{width = 0.70\textwidth}{figs/efficiency}
		% \begin{enumerate}[(1)]
		% 	\setlength{\itemsep}{15pt}
		% 	\item \red{\it Effective:}
		% 	  \gray{Can \polysi{} find SI violations in databases?}
		% 	\item \blue{\it Informative:}
		% 	  \gray{Can \polysi{} provide understandable counterexamples for SI violations?}
		% 	\item \purple{\boxed{\it Efficient:}}
		% 	  How efficient is \polysi?
		% \end{enumerate}
	\end{center}
\end{frame}
%%%%%%%%%%%%%%%%%%%%

%%%%%%%%%%%%%%%%%%%%
\begin{frame}{}
	\begin{center}
		{\input{tables/workload}}
	\end{center}
\end{frame}
%%%%%%%%%%%%%%%%%%%%

% %%%%%%%%%%%%%%%%%%%%
% \begin{frame}{}
% 	\begin{center}
% 		\begin{description}[GeneralRW:]
% 			\setlength{\itemsep}{10pt}
% 			\item[RuBiS:] an eBay-like bidding system
% 			\item[TPC-C:] an open standard for OLTP benchmarking
% 			\item[C-Twitter:] a Twitter clone
% 			\vspace{10pt}
% 			\item[GeneralRH:] read-heavy workloads with $95\%$ reads
% 			\item[GeneralRW:] medium workloads with $50\%$ reads
% 			\item[GeneralWH:] write-heavy workloads with $30\%$ reads
% 		\end{description}
% 	\end{center}
% \end{frame}
% %%%%%%%%%%%%%%%%%%%%

% %%%%%%%%%%%%%%%%%%%%
% \begin{frame}{Reproducing Known SI Violations}
% 	\begin{center}
% 		{\input{tables/effectiveness-reproduce}}
% 		\vspace{0.80cm}

% 		An extensive collection of 2477 anomalous histories \\[2pt]
% 		\ncite{Complexity:OOPSLA2019, CockroachDB-bug, YugabyteDB-bug}
% 	\end{center}
% \end{frame}
% %%%%%%%%%%%%%%%%%%%%

% %%%%%%%%%%%%%%%%%%%%
% \begin{frame}{Detecting New SI Violations}
% 	\begin{center}
% 		\red{Dgraph}: helped the Dgraph team \blue{confirm} some of their suspicions
% 		  about their latest release

% 		\vspace{0.50cm}
% 		{\input{tables/effectiveness-new}}
% 		\vspace{0.50cm}

% 		\red{Galera}: \blue{confirmed} the incorrect claim on preventing ``lost updates''
% 		  for transactions issued on different cluster nodes
% 	\end{center}
% \end{frame}
% %%%%%%%%%%%%%%%%%%%%

% %%%%%%%%%%%%%%%%%%%%
% \begin{frame}{Understanding Violations (Lost Update)}
% 	\begin{center}
% 		\input{figs/informativeness}
% 	\end{center}
% \end{frame}
% %%%%%%%%%%%%%%%%%%%%

% %%%%%%%%%%%%%%%%%%%%
% \begin{frame}{Performance Evaluation}
% 	\begin{description}
% 		\setlength{\itemsep}{15pt}
% 		\item[dbcop~\ncite{Complexity:OOPSLA2019}:]
% 			the state-of-the-art SI checker
% 		\item[CobraSI:] reducing SI checking to SER checking \\
% 		  \ncite{Complexity:OOPSLA2019} to leverage Cobra with/without GPU
% 			\vspace{0.20cm}
% 			\begin{description}
% 				\item[Cobra~\ncite{Cobra:OSDI2020}:]
% 					the state-of-the-art SER checker using both MonoSAT and GPU
% 			\end{description}
% 	\end{description}
% \end{frame}
% %%%%%%%%%%%%%%%%%%%%

%%%%%%%%%%%%%%%%%%%%
\begin{frame}{}
	\centerline{\polysi{} 的性能显著优于其它 SI 检测算法。\footnote{
		本实验使用由 PostgreSQL 产生的满足 SI 的执行历史。
	}}

	\fig{width = 0.50\textwidth}{figs/polysi-runtime}
	  % {Performance comparison under various workloads ({\it timeout = $180s$}).}
  %\#sessions=20, \#txns/session=100, \#ops/txn=15,   keys=10k,  \%read=50\%, distribution=zipfian.
\end{frame}
%%%%%%%%%%%%%%%%%%%%

%%%%%%%%%%%%%%%%%%%%
\begin{frame}{}
	\centerline{\polysi{} 消耗更少的内存。}
	\fig{width = 0.50\textwidth}{figs/polysi-memory}
  %\#sessions=20, \#txns/session=100, \#ops/txn=15,   keys=10k,  \%read=50\%, distribution=zipfian.
\end{frame}
%%%%%%%%%%%%%%%%%%%%

%%%%%%%%%%%%%%%%%%%%
\begin{frame}{}
	\begin{center}
		\polysi{} 可用于检测大规模执行历史: 1M 事务, 1B 数据项

		\vspace{0.30cm}
		\fig{width = 0.70\textwidth}{figs/polysi-scalability}
		\vspace{0.30cm}
	\end{center}
\end{frame}
%%%%%%%%%%%%%%%%%%%%

%%%%%%%%%%%%%%%%%%%%
\begin{frame}{}
	\begin{center}
		\blue{剪枝 (P)} 对 \polysi{} 的性能影响较大。\footnote{
			\blue{紧凑编码 (C)} 从略。
		}

		\vspace{0.30cm}
		\fig{width = 0.70\textwidth}{figs/polysi-diff}
	\end{center}
\end{frame}
%%%%%%%%%%%%%%%%%%%%

%%%%%%%%%%%%%%%%%%%%
\begin{frame}{}
	\begin{center}
		在实验中, 剪枝步骤可以删除绝大多数约束。

		\vspace{0.30cm}
		\input{tables/pruning}
		\vspace{0.30cm}

		\red{TPC-C}: 仅包含只读事务与 RMW 事务
	\end{center}
\end{frame}
%%%%%%%%%%%%%%%%%%%%