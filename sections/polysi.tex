% polysi.tex

%%%%%%%%%%%%%%%%%%%%
\begin{frame}{Database Systems and Snapshot Isolation}
  \begin{center}
    Many database systems implement snapshot isolation.

    \vspace{0.30cm}
    \fig{width = 0.80\textwidth}{figs/db-si}
  \end{center}
\end{frame}
%%%%%%%%%%%%%%%%%%%%

%%%%%%%%%%%%%%%%%%%%
\begin{frame}{Database Systems and Snapshot Isolation}
  \begin{center}
    Database systems may \red{fail} to provide snapshot isolation correctly.

    \vspace{0.30cm}
    \fig{width = 0.85\textwidth}{figs/db-si-violations}
  \end{center}
\end{frame}
%%%%%%%%%%%%%%%%%%%%

%%%%%%%%%%%%%%%%%%%%
\begin{frame}{The SI Checking Problem}
  \begin{definition}[The SI Checking Problem]
    The SI checking problem is the \purple{decision problem} of determing \\[5pt]
    whether a \teal{{\it history} $\H = (T, \SO)$}\footnote{
      We take the common ``UniqueValue'' assumption on histories:
      for each key, every write to the key assigns a unique value.
    } of a database system satisfies SI?
  \end{definition}

  \fig{width = 0.60\textwidth}{figs/si-checking}

  \vspace{-0.50cm}
  \[
    \SO: \text{{\it session order} among the set $T$ of transactions}
  \]
\end{frame}
%%%%%%%%%%%%%%%%%%%%

%%%%%%%%%%%%%%%%%%%%
\begin{frame}{The SI Checking Problem}
  \begin{center}
    \uncover<1->{
      \blue{\it Sound:} If the checker says \no,
        then the history does {\it not} satisfy SI.
    }

    \vspace{0.20cm}
    \uncover<2->{
      \blue{\it Complete:} If the checker says \yes,
        then the history {\it satisfies} SI.
    }

    \vspace{0.30cm}
    \input{tikz/checker-tikz}
    \vspace{0.30cm}

    \uncover<3->{
      \blue{\it Efficient:} The checker should {\it scale} up to large workloads.
    }

    \vspace{0.20cm}
    \uncover<4>{
      \blue{\it Informative:} The checker should provide
        understandable {\it counterexamples} if it finds violations.
    }
  \end{center}
\end{frame}
%%%%%%%%%%%%%%%%%%%%

%%%%%%%%%%%%%%%%%%%%
\begin{frame}{Related Work}
  \begin{description}
    \setlength{\itemsep}{15pt}
    \item[dbcop~\ncite{Complexity:OOPSLA2019}] checker for SI \\[2pt]
      not practically efficient; \\[2pt]
      not informative, returning only ``\textsf{False}'' upon violations
    \item[Cobra~\ncite{Cobra:OSDI2020}] state-of-the-art checker for SER \\[2pt]
      The SI checking problem is {\it harder}.
  \end{description}
\end{frame}
%%%%%%%%%%%%%%%%%%%%

%%%%%%%%%%%%%%%%%%%%
\begin{frame}{Related Work}
  \begin{description}
    \item[Elle~\ncite{Elle:VLDB2020}] checker for various isolation levels \\[2pt]

      \vspace{0.20cm}
      Work perfectly on traceable and recoverable histories;
      but may be incomplete on the key-value datatype

      \pause
      \vspace{0.20cm}
      SI checking based on the Adya-style notions~\ncite{Adya:PhDThesis1999}
      relies on the start/commit timestamps of transactions.\\[2pt]

      \pause
      \vspace{0.20cm}
      SI checking based on the~\ncite{AnalysingSI:JACM2018} notions
      may miss SI violations.\footnote{
        Could Elle tell the difference between snapshot isolation and strong-session-snapshot-isolation? https://github.com/jepsen-io/elle/issues/17}
        \footnote{Elle may miss two types of transaction anomalies. https://github.com/jepsen-io/elle/issues/21}
  \end{description}
\end{frame}
%%%%%%%%%%%%%%%%%%%%

%%%%%%%%%%%%%%%%%%%%
\begin{frame}{Contribution: the \polysi{} Checker}
  \fig{width = 0.70\textwidth}{figs/checker-polysi}
\end{frame}
%%%%%%%%%%%%%%%%%%%%

%%%%%%%%%%%%%%%%%%%%
\begin{frame}{Contribution: the \polysi{} Checker}
  \begin{center}
    \input{tikz/polysi-checker-tikz}

    \vspace{0.50cm}
    \only<1>{
      \blue{{\it Sound} \&{\it Complete:}}
        a novel {\it polygraph} based characterization of SI
    }
    \only<2>{
      \blue{\it Efficient:} utilizing MonoSAT solver optimized for graph problems
    }
    \only<3>{
      \blue{\it Efficient:} pruning the constraints on {\it polygraph} before encoding
    }
    \only<4>{
      \blue{\it Informative:} extract counterexamples from the UNSAT core
    }
  \end{center}
\end{frame}
%%%%%%%%%%%%%%%%%%%%

%%%%%%%%%%%%%%%%%%%%
\begin{frame}{\polysi: Polygraph based Characterization of SI}
	\begin{center}
		% \fig{width = 0.90\textwidth}{figs/polysi-checker-polygraph}
		% \vspace{0.30cm}
		Before this, we first review the {\it dependency graph} based \\[5pt]
		characterization of SI~\ncite{AnalysingSI:JACM2018}.

		\vspace{0.30cm}
		\begin{theorem}[Theorem 4.1 of~\ncite{AnalysingSI:JACM2018}]
			Informally, a history satisfies SI if and only if \\[3pt]
			\teal{there exists} a \red{dependency graph} for it that \\[3pt]
			contains only cycles (if any) with \blue{at least two adjacent $\RW$} edges.
		\end{theorem}
	\end{center}
\end{frame}
%%%%%%%%%%%%%%%%%%%%

%%%%%%%%%%%%%%%%%%%%
\begin{frame}{Dependency Graph based Characterization of SI}
  \begin{center}
		\only<4>{
			% $T_{0} \rel{\WR} T_{A} \land T_{0} \rel{\WW} T_{B}
			%   \implies T_{A} \rel{\RW} T_{B}$
			$T_{A}$ reads from $T_{0}$ which is overwritten by $T_{B}$
		}
		\only<5>{
			% $T_{0} \rel{\WR} T_{B} \land T_{0} \rel{\WW} T_{A}
			%   \implies T_{A} \rel{\RW} T_{A}$
			$T_{B}$ reads from $T_{0}$ which is overwritten by $T_{A}$
		}
		\only<7->{
			\red{$\boxed{\text{\it Suppose that}\;\; T_{A} \rel{\WW} T_{B}}$}
		}

		\vspace{0.20cm}
		{\input{tikz/banking-lost-update-depgraph-tikz}}
		\vspace{0.20cm}

		\only<2>{
			$\WR$: ``write-read'' dependency capturing the ``read-from'' relation
		}
		\only<3>{
			$\WW$: ``write-write'' dependency capturing the version order on $\acct$
		}
		\only<4-5>{
			$\RW$: ``read-write'' dependency
		}
		\only<6>{
			The cycle $T_{A} \rel{\RW} T_{B} \rel{\RW} T_{A}$
			is \blue{allowed} by SI.
		}
		\only<7>{
			undesired cycle for SI: $T_{A} \rel{\WW} T_{B} \rel{\RW} T_{A}$
		}
  \end{center}
\end{frame}
%%%%%%%%%%%%%%%%%%%%

%%%%%%%%%%%%%%%%%%%%
\begin{frame}{Dependency Graph based Characterization of SI}
	\begin{center}
		\uncover<2->{
			\red{$\boxed{\text{\it Suppose that}\;\; T_{B} \rel{\WW} T_{A}}$}
		}

		\vspace{0.20cm}
		{\input{tikz/banking-lost-update-depgraph-ww-tbta-tikz}}
		\vspace{0.20cm}

		\uncover<2>{
			undesired cycle for SI: $T_{B} \rel{\WW} T_{A} \rel{\RW} T_{B}$
		}
	\end{center}
\end{frame}
%%%%%%%%%%%%%%%%%%%%

%%%%%%%%%%%%%%%%%%%%
\begin{frame}{Dependency Graph based Characterization of SI}
  \begin{center}
		We have considered both bases $T_{A} \rel{\WW} T_{B}$
		and $T_{B} \rel{\WW} T_{A}$, \\[2pt]
		and each case leads to an undesired cycle for SI.

		\vspace{0.20cm}
		\fig{width = 0.65\textwidth}{figs/banking-lost-update-wr}
		\vspace{0.20cm}

		Therefore, this history does not satisfy SI.
  \end{center}
\end{frame}
%%%%%%%%%%%%%%%%%%%%

%%%%%%%%%%%%%%%%%%%%
\begin{frame}{Dependency Graph based Characterization of SI}
	\begin{theorem}[Equivalence of Theorem 4.1 of~\ncite{AnalysingSI:JACM2018}]
		Informally, a history satisfies SI if and only if \\[3pt]
		\teal{there exists} a \red{dependency graph} $\G$ for it such that \\[3pt]
		\cyan{the induced graph of $\G$} ${\boxed{((\SO_{\G} \cup \WR_{\G} \cup \WW_{\G}) \comp \RW_{\G}?)}}$ \text{\it is acyclic}.
	\end{theorem}
\end{frame}
%%%%%%%%%%%%%%%%%%%%

%%%%%%%%%%%%%%%%%%%%
\begin{frame}{Dependency Graph based Characterization of SI}
	\[
		\text{\it induced graph}\; {\boxed{((\SO_{\G} \cup \WR_{\G} \cup \WW_{\G}) \comp \RW_{\G}?)}} \text{\it\; for $\G$}
	\]

	\begin{center}
		\resizebox{0.60\textwidth}{!}{\input{tikz/banking-lost-update-depgraph-theorem-tikz.tex}}
		\vspace{0.30cm}

		\uncover<2->{
			first composing ($\comp$) the $\SO$/$\WR$/$\WW$ edges with the $\RW$ edges
		}

		\vspace{0.20cm}
		\uncover<3>{
			then deleting all the $\RW$ edges
		}
	\end{center}
\end{frame}
%%%%%%%%%%%%%%%%%%%%

%%%%%%%%%%%%%%%%%%%%
\begin{frame}{Polygraph: A Family of Dependency Graphs}
	\begin{center}
		Consider the two cases of $\WW$ dependencies between $T_{A}$ and $T_{B}$.
	\end{center}

	\vspace{-0.20cm}
	\begin{columns}[c]
		\column{0.50\textwidth}
			\fig{width = 0.80\textwidth}{figs/banking-lost-update-depgraph}
		\column{0.50\textwidth}
			\fig{width = 0.80\textwidth}{figs/banking-lost-update-depgraph-ww-tbta}
	\end{columns}

	\vspace{-0.20cm}
	\begin{center}
		\pause
		\fig{width = 0.50\textwidth}{figs/banking-lost-update-polygraph}
		polygraph:
		$\tuple{\cyan{\eithervar} \triangleq \set{T_{A} \rel{\WW} T_{B}},
				\cyan{\orvar} \triangleq \set{T_{B} \rel{\WW} T_{A}, T_{A}' \rel{\RW} T_{A}}}
		$
	\end{center}
\end{frame}
%%%%%%%%%%%%%%%%%%%%

%%%%%%%%%%%%%%%%%%%%
\begin{frame}{\textsc{PolySI}: A Running Example}
	\begin{center}
		To explain the whole \polysi{} procedure with a running example.

		\vspace{0.50cm}
		\fig{width = 0.90\textwidth}{figs/polysi-checker-pruning-encoding-solving}
	\end{center}
\end{frame}
%%%%%%%%%%%%%%%%%%%%

%%%%%%%%%%%%%%%%%%%%
\begin{frame}{\textsc{PolySI}: A Running Example}
	\begin{center}
		\resizebox{0.90\textwidth}{!}{\input{tikz/polysi-alg-tikz}}

		\only<1>{
			$\WW$ between $T_{0}$, $T_{1}$, and $T_{5}$ (on $\keyxvar$)
			and between $T_{0}$ and $T_{2}$ (on $\keyyvar$)
		}

		\only<2>{
			The $T_{5} \rel{\WW(\keyxvar)} T_{0}$ case is pruned
			due to $T_{0} \rel{\SO} T_{5} \rel{\WW(\keyxvar)} T_{0}$.
		}

		\only<4>{
			The $T_{0} \rel{\WW(\keyxvar)} T_{5}$ case becomes known.
		}

		\only<6>{
			The $T_{1} \rel{\WW(\keyxvar)} T_{0}$ case is pruned
			due to $T_{3} \rel{\RW(\keyxvar)} T_{0} \rel{\WR(\keyyvar)} T_{3}$.
		}

		\only<8>{
			The $T_{0} \rel{\WW(\keyxvar)} T_{1}$ case becomes known.
		}

		\only<10>{
			The $T_{2} \rel{\WW(\keyyvar)} T_{0}$ case is pruned, \\
			while the $T_{0} \rel{\WW(\keyyvar)} T_{2}$ case becomes known.
		}
		\only<12>{
			The $\WW$ order between $T_{1}$ and $T_{5}$ is still uncertain after pruning.
		}
	\end{center}
\end{frame}
%%%%%%%%%%%%%%%%%%%%

%%%%%%%%%%%%%%%%%%%%
\begin{frame}{\textsc{PolySI}: A Running Example}
	\vspace{-0.50cm}
	\[\tuple{
		\uncover<2->{\purple{\eithervar} = \set{T_{1} \rel{\WW(\keyxvar)} T_{5},
			T_{3} \rel{\RW(\keyxvar)} T_{5}}},
		\uncover<2->{\violet{\orvar} = \set{T_{5} \rel{\WW(\keyxvar)} T_{1}}}
	}\]

	\vspace{-0.30cm}
	\begin{center}
		\resizebox{0.80\textwidth}{!}{\input{tikz/polysi-alg-encoding-tikz}}
	\end{center}
	\vspace{-0.80cm}

	\uncover<3->{
		\[
			\underbrace{\purple{(\BV_{1,5} \land \BV_{3,5} \land \lnot \BV_{5,1})}}_{\purple{\eithervar}} \lor
			\underbrace{\violet{(\BV_{5,1} \land \lnot \BV_{1,5} \land \lnot \BV_{3,5})}}_{\violet{\orvar}}
		\]
	}
\end{frame}
%%%%%%%%%%%%%%%%%%%%

%%%%%%%%%%%%%%%%%%%%
\begin{frame}{\textsc{PolySI}: A Running Example}
	\vspace{-0.50cm}
	\[
		\cyan{\text{induced graph}}\; \cyan{\mathcal{I}}\;
		  {\boxed{((\SO_{\G} \cup \WR_{\G} \cup \WW_{\G}) \comp \RW_{\G}?)}}
	\]

  \begin{columns}
		\column{0.50\textwidth}
			\fig{width = 1.00\textwidth}{figs/polysi-alg-final}
		\column{0.50\textwidth}
			\fig{width = 1.00\textwidth}{figs/polysi-alg-encoding}
	\end{columns}

	\vspace{0.50cm}
	\uncover<2->{
	\[
		T_{1} \rel{\WR} T_{3} \rel{\RW} T_{5}:\;
		  \BV_{1,5}^{\cyan{\;\mathcal{I}}} = \BV_{1,3} \;\land\; \BV_{3,5} \quad
	\]

	\vspace{-0.20cm}
	\begin{center}
		The presence of the edge $T_{1} \to T_{5}$ in the induced graph $\mathcal{I}$ \\[2pt]
		depends on that of the edges $T_{1} \to T_{3}$ and $T_{3} \to T_{5}$ in the polygraph.
	\end{center}
	}
	% \vspace{-0.30cm}
	% \uncover<2->{
	% \[
	% 	T_{1} \rel{\WR} T_{3} \rel{\RW} T_{2}:\;
	% 	  \BV_{1,2}^{\cyan{\;\mathcal{I}}} = \BV_{1,3} \;\land\; \BV_{3,2} \quad
	% \]
	% }
\end{frame}
%%%%%%%%%%%%%%%%%%%%

%%%%%%%%%%%%%%%%%%%%
\begin{frame}{\textsc{PolySI}: A Running Example}
	\begin{center}
		Feed the SAT formula into the \blue{MonoSAT} solver~\ncite{MonoSAT:AAAI2015} \\[2pt]
		which is optimized for \purple{\it cycle detection}.

		\vspace{0.20cm}
		\fig{width = 0.40\textwidth}{figs/sat-solver}
	\end{center}
\end{frame}
%%%%%%%%%%%%%%%%%%%%

%%%%%%%%%%%%%%%%%%%%
\begin{frame}{\textsc{PolySI}: A Running Example}
	\begin{center}
		\fig{width = 0.50\textwidth}{figs/polysi-alg-cycle}

		\vspace{0.20cm}
		The MonoSAT solver finds an undesired cycle for SI.
	\end{center}
\end{frame}
%%%%%%%%%%%%%%%%%%%%

%%%%%%%%%%%%%%%%%%%%
\begin{frame}{Experimental Evaluation}
	\begin{center}
		\begin{enumerate}[(1)]
			\setlength{\itemsep}{15pt}
			\item \red{\it Effective:}
			  Can \polysi{} find SI violations in databases?
			\item \blue{\it Informative:}
			  Can \polysi{} provide understandable counterexamples for SI violations?
			\item \purple{\it Efficient:}
			  How efficient is \polysi?
		\end{enumerate}

		% \vspace{0.50cm}
		% \url{https://github.com/hengxin/PolySI-PVLDB2023-Artifacts}
	\end{center}
\end{frame}
%%%%%%%%%%%%%%%%%%%%

%%%%%%%%%%%%%%%%%%%%
\begin{frame}{Workloads}
	\begin{center}
		{\input{tables/workload}}
	\end{center}
\end{frame}
%%%%%%%%%%%%%%%%%%%%

%%%%%%%%%%%%%%%%%%%%
\begin{frame}{Benchmarks}
	\begin{center}
		\begin{description}[GeneralRW:]
			\setlength{\itemsep}{10pt}
			\item[RuBiS:] an eBay-like bidding system
			\item[TPC-C:] an open standard for OLTP benchmarking
			\item[C-Twitter:] a Twitter clone
			\vspace{10pt}
			\item[GeneralRH:] read-heavy workloads with $95\%$ reads
			\item[GeneralRW:] medium workloads with $50\%$ reads
			\item[GeneralWH:] write-heavy workloads with $30\%$ reads
		\end{description}
	\end{center}
\end{frame}
%%%%%%%%%%%%%%%%%%%%

%%%%%%%%%%%%%%%%%%%%
\begin{frame}{Reproducing Known SI Violations}
	\begin{center}
		{\input{tables/effectiveness-reproduce}}
		\vspace{0.80cm}

		An extensive collection of 2477 anomalous histories \\[2pt]
		\ncite{Complexity:OOPSLA2019, CockroachDB-bug, YugabyteDB-bug}
	\end{center}
\end{frame}
%%%%%%%%%%%%%%%%%%%%

%%%%%%%%%%%%%%%%%%%%
\begin{frame}{Detecting New SI Violations}
	\begin{center}
		\red{Dgraph}: helped the Dgraph team \blue{confirm} some of their suspicions
		  about their latest release

		\vspace{0.50cm}
		{\input{tables/effectiveness-new}}
		\vspace{0.50cm}

		\red{Galera}: \blue{confirmed} the incorrect claim on preventing ``lost updates''
		  for transactions issued on different cluster nodes
	\end{center}
\end{frame}
%%%%%%%%%%%%%%%%%%%%

%%%%%%%%%%%%%%%%%%%%
\begin{frame}{Understanding Violations (Lost Update)}
	\begin{center}
		\input{figs/informativeness}
	\end{center}
\end{frame}
%%%%%%%%%%%%%%%%%%%%

%%%%%%%%%%%%%%%%%%%%
\begin{frame}{Performance Evaluation}
	\begin{description}
		\setlength{\itemsep}{15pt}
		\item[dbcop~\ncite{Complexity:OOPSLA2019}:]
			the state-of-the-art SI checker
		\item[CobraSI:] reducing SI checking to SER checking \\
		  \ncite{Complexity:OOPSLA2019} to leverage Cobra with/without GPU
			\vspace{0.20cm}
			\begin{description}
				\item[Cobra~\ncite{Cobra:OSDI2020}:]
					the state-of-the-art SER checker using both MonoSAT and GPU
			\end{description}
	\end{description}
\end{frame}
%%%%%%%%%%%%%%%%%%%%

%%%%%%%%%%%%%%%%%%%%
\begin{frame}{Performance Evaluation: Runtime}
	\centerline{\polysi{} significantly outperforms the competitors.\footnote{
		All the input histories extracted from PostgreSQL satisfy SI.
	}}

	\fig{width = 0.50\textwidth}{figs/polysi-runtime}
	  % {Performance comparison under various workloads ({\it timeout = $180s$}).}
  %\#sessions=20, \#txns/session=100, \#ops/txn=15,   keys=10k,  \%read=50\%, distribution=zipfian.
\end{frame}
%%%%%%%%%%%%%%%%%%%%

%%%%%%%%%%%%%%%%%%%%
\begin{frame}{Performance Evaluation: Memory}
	\centerline{\polysi{} consumes less memory.}
	\fig{width = 0.50\textwidth}{figs/polysi-memory}
  %\#sessions=20, \#txns/session=100, \#ops/txn=15,   keys=10k,  \%read=50\%, distribution=zipfian.
\end{frame}
%%%%%%%%%%%%%%%%%%%%

%%%%%%%%%%%%%%%%%%%%
\begin{frame}{Performance Evaluation: Scalability}
	\begin{center}
		several hours and $35 \sim 40$GB memory for checking \blue{1M} transactions

		\vspace{0.30cm}
		\fig{width = 0.70\textwidth}{figs/polysi-scalability}
		\vspace{0.30cm}

		large workloads: 1B keys and 1M transactions, long transactions
	\end{center}
\end{frame}
%%%%%%%%%%%%%%%%%%%%

%%%%%%%%%%%%%%%%%%%%
\begin{frame}{Performance Evaluation: Differential Analysis}
	\begin{center}
		\blue{Pruning (P)} is crucial to the efficiency of \polysi.\footnote{
			\blue{Compacting (C)} encoding has been omitted in this presentation.
		}

		\vspace{0.30cm}
		\fig{width = 0.70\textwidth}{figs/polysi-diff}
	\end{center}
\end{frame}
%%%%%%%%%%%%%%%%%%%%

%%%%%%%%%%%%%%%%%%%%
\begin{frame}{Performance Evaluation: Pruning}
	\begin{center}
		\polysi{} can \blue{prune} a huge number of constraints before encoding.

		\vspace{0.30cm}
		\input{tables/pruning}
		\vspace{0.30cm}

		\red{TPC-C}: read-only transactions + RMW transactions
	\end{center}
\end{frame}
%%%%%%%%%%%%%%%%%%%%