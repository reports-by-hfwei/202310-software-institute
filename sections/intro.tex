% intro.tex

%%%%%%%%%%%%%%%%%%%%
\begin{frame}{Transaction and Isolation Level}
  \begin{center}
    A transaction is a \blue{\it group} of operations
    that are executed \red{atomically}.

    \vspace{0.30cm}
		\resizebox{0.55\textwidth}{!}{\input{tikz/isolation-intro-write-skew-tikz}}

    \vspace{0.20cm}
    The isolation levels specify how concurrent transactions \\[2pt]
    are isolated from each other.
  \end{center}
\end{frame}
%%%%%%%%%%%%%%%%%%%%

%%%%%%%%%%%%%%%%%%%%
\begin{frame}{Serializability (SER)}
  \begin{center}
    All transactions appear to be executed in some total order.

    \vspace{0.30cm}
		\resizebox{0.50\textwidth}{!}{\input{tikz/isolation-ser-write-skew-tikz}}

    \vspace{0.20cm}
    However, implementing serializability is too expensive.
  \end{center}
\end{frame}
%%%%%%%%%%%%%%%%%%%%

%%%%%%%%%%%%%%%%%%%%
\begin{frame}{Snapshot Isolation (SI)}
  \begin{center}
		\resizebox{0.50\textwidth}{!}{\input{tikz/isolation-si-write-skew-tikz}}

    \vspace{0.20cm}
    \violet{Snapshot Read:} Each transaction reads data from a {\it snapshot} \\
      as of the time the transaction started.
  \end{center}
\end{frame}
%%%%%%%%%%%%%%%%%%%%

%%%%%%%%%%%%%%%%%%%%
\begin{frame}{Snapshot Isolation (SI)}
  \begin{center}
    \resizebox{0.48\textwidth}{!}{\input{tikz/isolation-si-lost-update-tikz}}
  \end{center}

  \vspace{-0.50cm}
  \violet{Snapshot Write:}
    Concurrent transactions {\it cannot} write to the same key.
    One of them must be aborted.
\end{frame}
%%%%%%%%%%%%%%%%%%%%