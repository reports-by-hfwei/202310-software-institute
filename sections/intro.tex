% intro.tex

%%%%%%%%%%%%%%%%%%%%
\begin{frame}{}
	\fig{width = 0.70\textwidth}{figs/verifying-history}
\end{frame}
%%%%%%%%%%%%%%%%%%%%

%%%%%%%%%%%%%%%%%%%%
\begin{frame}{}
  \begin{center}
    可序列化 (Serializability): 所有事务就像按照某个全序执行一样。

    \vspace{0.30cm}
    \fig{width = 0.70\textwidth}{figs/cobra-osdi2020}
  \end{center}
\end{frame}
%%%%%%%%%%%%%%%%%%%%

%%%%%%%%%%%%%%%%%%%%
\begin{frame}{}
  \begin{center}
    为了提高性能, 很多数据库仅提供快照隔离 (Snapshot Isolation)。

    \vspace{0.30cm}
    \fig{width = 0.80\textwidth}{figs/db-si}
  \end{center}
\end{frame}
%%%%%%%%%%%%%%%%%%%%

%%%%%%%%%%%%%%%%%%%%
\begin{frame}{Snapshot Isolation (SI)}
  \begin{center}
		\resizebox{0.50\textwidth}{!}{\input{tikz/isolation-si-write-skew-tikz}}

    \vspace{0.20cm}
    \violet{Snapshot Read:} Each transaction reads data from a {\it snapshot} \\
      as of the time the transaction started.
  \end{center}
\end{frame}
%%%%%%%%%%%%%%%%%%%%

%%%%%%%%%%%%%%%%%%%%
\begin{frame}{Snapshot Isolation (SI)}
  \begin{center}
    \resizebox{0.48\textwidth}{!}{\input{tikz/isolation-si-lost-update-tikz}}
  \end{center}

  \vspace{-0.50cm}
  \violet{Snapshot Write:}
    Concurrent transactions {\it cannot} write to the same key.
    One of them must be aborted.
\end{frame}
%%%%%%%%%%%%%%%%%%%%

%%%%%%%%%%%%%%%%%%%%
\begin{frame}{}
  \begin{center}
    但是, 很多数据库\red{\bf 未能}正确实现快照隔离。

    \vspace{0.30cm}
    \fig{width = 0.85\textwidth}{figs/db-si-violations}
  \end{center}
\end{frame}
%%%%%%%%%%%%%%%%%%%%

%%%%%%%%%%%%%%%%%%%%
\begin{frame}{}
  \begin{center}
    高效的快照隔离检测算法

    \fig{width = 0.85\textwidth}{figs/polysi-vldb2023}
  \end{center}
\end{frame}
%%%%%%%%%%%%%%%%%%%%

%%%%%%%%%%%%%%%%%%%%
\begin{frame}{}
  \begin{center}
    \fig{width = 0.70\textwidth}{figs/checker-polysi}
  \end{center}
\end{frame}
%%%%%%%%%%%%%%%%%%%%